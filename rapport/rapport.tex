\documentclass[a4paper]{article}

%% Language and font encodings
\usepackage[french]{babel}
\usepackage[utf8x]{inputenc}
\usepackage[T1]{fontenc}
\usepackage{graphicx}
\usepackage{subcaption}

%% pour le code
\usepackage{listings}
\usepackage{color}

%% Sets page size and margins
\usepackage[a4paper,top=3cm,bottom=2cm,left=3cm,right=3cm,marginparwidth=1.75cm]{geometry}

%% Useful packages
\usepackage{amsmath,amsthm,amsfonts,amssymb, stmaryrd}
\usepackage{graphicx}
\usepackage[colorinlistoftodos]{todonotes}
\usepackage[colorlinks=true, allcolors=blue]{hyperref}
\usepackage{xcolor}


\title{Projet d'informatique : Inégalités linéaires et vérifications de programmes}
\author{Gustave Billon, François Hublet}

\begin{document}
\maketitle

\section{Organisation du projet}

Le projet comporte cinq parties principales : le corps du programme se trouve dans le fichier linlang.ml, qui s'appuie sur un lexer (lexer.mll), un parser (parser.mly), l'algorithme du simplexe (simplex.ml) et l'algorithme de Fourier-Motzkin

\section{Structures de données}

On a choisi de représenter les combinaisons linéaires à l'aide de fractions, définies dans le module Fraction, afin de mener les calculs de façon purement algébrique. Les inégalités étant beaucoup sujettes à des manipulations d'algèbre linéaire, elles sont représentées dans des Fraction.frac array, et l'algorithme du simplexe est codé de manière impérative, ce qui se prête mieux aux calculs matriciels.

\end{document}
